\documentclass[11pt, a4paper]{article}
\usepackage{color}
\usepackage{listings}
\usepackage{fullpage}
\usepackage{amsfonts}
\usepackage[latin1]{inputenc}
\usepackage[english]{babel}
\usepackage[T1]{fontenc}
\usepackage{graphicx}
\usepackage{amsmath,amssymb}
\usepackage{graphicx}
\usepackage{fancyhdr}           %Min header begynder
\pagestyle{fancy}
\fancyhf{}
\setlength{\headheight}{14pt}
\setlength{\headsep}{25pt}
\fancyhead[L]{Connect Four}  %Header venstre
\fancyhead[C]{Group I}			 %header center
\fancyhead[R]{ISP 2012, ITU} 		 %header h�jre
\fancyfoot[R]{\thepage} 				 %Sidetal i bunden
\renewcommand\headrulewidth{0.4pt}
\renewcommand\footrulewidth{0.4pt}

\author{Mark Gray, Johan Sivertsen \& Lauge Groes}
\title{Intelligent Systems Programming, Assignment 1: Connect Four Game}
\date{\today}
\begin{document}

\lstset{language=Java, breaklines = true, identifierstyle=\ttfamily,
keywordstyle=\color[rgb]{0.700,0.100,0.500},
commentstyle=\color[rgb]{0.133,0.545,0.133},
stringstyle=\color[rgb]{0,0,1}, extendedchars=true,
basicstyle=\scriptsize,numbers=left,numberstyle=\tiny}
\maketitle
\newpage
\section*{Introduction}


We have developed an implementation of a connect four agent that uses a
combination of the minimax algorithm, depth-limited search,
\alpha-\beta-pruning and a heuristic evaluation function.

\section*{Search and Cut-off Function}
Our first attempt was a simple depth-first search-implementation of the
mini-max algorithm, which included a cut-off after a certain number recursive
calls and \alpha - \beta-pruning. 

In the the depth-first search-implementation, we would recursively consider the
opponents response to each action. This amounted to recursively calling the min
and max methods for each action associated with the state we were in at the time
of the call. Our experience led us to introduce some changes to the final AI
player.

The depth-first search led to only a small part of the search space to
evaluated since we would sequantially start with the first action of the first
state considereder and then first action the resulting and hence forward until
we reached a terminal state. We would only reach a few terminal states when the
board size was large enough, due to our time constraints.

Even when \alpha\-\beta\-pruning was introduced, the search space was still too
large for board sizes larger than 4x4. 

The solution was to introduce depth-limited search and a heuristic evaluation
function. Introducing depth-limited search amounted to changed how we our
cut-offs worked (i.e. not cut-offs due to \alpha-\beta-pruning). Instead of a
counter that increased with every recursive call, we introduced a counter for
the depth of the search tree.

Without a heuristic the changes made the new AI player worse for larger boards,
when we kept the depth of the search tree on the lower end. Since no terminal
state was ever reached, the utility of playing a certain action was never
evaluated. This shows that a heuristic is extremely important but also that it
is the heuristic that decides moves, and thus the logic, for the AI player,
until it can reach terminal states in its search. Choosing a good heuristic is
thus what this approach boils down to.

%% Eksempel til at sætte kode ind i tex. Kræver præcis linjeangivelse så lad os
% vente med det til sidst!

\begin{figure}
\lstinputlisting[firstline=1,lastline=56,firstnumber=1]{../src/Board.java}
\caption{ Excerpt of code}\label{label1}
\end{figure}


\section*{Evaluation Function}






\end{document}

